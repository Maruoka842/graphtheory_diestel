\subsection{}
まず,強連結な向き付けを持つならば $2$ 辺連結であることを示す.$2$ 辺連結でないとして,強連結な向き付けがないことをいえばよい.
辺 $e$ であって,$G-e$ が不連結なものをとる.$G-e$ の連結成分を $V_1$, $V_2$ とすると,$e$ の向きに応じて $V_1$ から $V_2$ または $V_2$ から $V_1$ への有向歩道が存在しないのでよい.

次に,$2$ 辺連結ならば強連結な向き付けが存在することを示す.適当な根 $r$ および normal spanning tree $T$ をひとつとり固定する.
$e = xy \in E(G)$ に対して,$x,y$ は$T$ の順序で比較可能である.$x\leq y$ としたとき,$e\in T$ ならば $x$ から $y$,$e\notin T$ ならば $y$ から $x$ へ向きを付ける.

この向き付けが強連結性を満たすことを示そう.任意に頂点 $x\neq r$ をとる.$r$ から $x$ および $x$ から $r$ への有向歩道が存在することを示せばよい.
$r$ から $x$ へは,$rTx$ が条件を満たす.$x$ から $r$ への有向歩道が存在しないとする.$x$ から有向歩道で到達可能な頂点のうち,木順序に関する極小元 $y$ をとる.
$y\neq r$ として矛盾を導く.

$y$ の親を $z$ とする($y\neq r$ より一意存在)と,辺連結性の仮定より $G-yz$ は連結.よって $y$ の部分木から部分木の外への $yz$ でない辺が存在する.
$a$ を部分木の点,$b$ を部分木外の点とすると,DFS木の性質より $b$ は $a$ の先祖なので $b \leq z < y \leq a$.すると $x$ から $z$ への有向歩道ができる $(x \to y\to a\to b\to z)$ ので $y$ の極小性に矛盾する.
